\chapter{Program Narrative}

%% PROGRAM NARRATIVE: The Program Narrative should summarize details of the
%% program over the performance period.  The specific content of the narrative is
%% not prescribed, but should instead focus on providing sufficient information
%% to judge the progress of the program with respect to the review criteria
%% described in Attachment 2.  It should be written concisely and convey all
%% information at a relatively high level.  Where possible this report should not
%% duplicate information available in publications, and should instead cite
%% reference as appropriate.

%% Additionally, programs should expect that reviewers are familiar with
%% challenges in the field and the report should only contain enough introductory
%% material as in necessary to justify the tasks and objectives presented in the
%% narrative.  The bulk of the narrative should focus on providing a clear
%% statement of the specific tasks, themes, and objectives of the program, and
%% succinctly highlight accomplishments made under the performance period with
%% regards to these items.  A timeline for the major activities and
%% accomplishments of the program over the performance period should also be
%% included.

%% The project narrative MUST NOT EXCEED 10 PAGES IN LENGHT, including charts,
%% graphs, maps, photographs, and other pictorial presentations, when printed
%% using standard 8.5” by 11” paper with 1 inch margins (top, bottom, left, and
%% right). The font must not be smaller than 11 point.

%% 1. Scientific and/or Technical Merit of the Project 
%% - What was the scientific/technical merit of the program? 
%% - What were the scientific/technical goals of the program? Are they sufficiently
%%   ambitious and has meaningful progress been made towards those goals during the
%%   performance period?
%% - How did the program compare with other research in its field in terms of
%%   scientific/technical merit, originality, and uniqueness? 
%% 2. Appropriateness of the Proposed Method or Approach 
%% - How logical and feasible were the research approaches? 
%% - Were the conceptual framework, methods, and analyses well justified, adequately
%%   developed, and comprehensive enough to produce scientifically valid conclusions? 
%% - How did the program approach potential problems, mitigate risks, and consider
%%   alternative strategies? 
%% 3. Competency of Personnel and Adequacy of Resources 
%% - How well qualified were the personnel to carry out the research?
%% - Were the resources (personnel, facilities, and budget) adequate for performing the
%%   research scope?
%% - Were the program leadership and resource management effective and efficient?
%% 4. Relevance and Impact
%% - How did the program contribute to the mission of the FES program during the
%%   performance period?
%% - How has the program impacted the field of fusion science and technology during the
%%   performance period?
%% - Has the program adequately disseminated scientific results through publications in
%%   peer-reviewed journals, meetings and conference presentations, workshops, or other
%%   appropriate means?
%% - Are there additional areas of impact that should be considered when evaluating this %%   program (e.g. educational benefits, contributions to other scientific fields,
%%   leadership activities, etc.)?


\section{Introduction}

The fundamental objective of the \gls{UW-FTI}'s fusion nuclear science and
engineering program is based upon providing nuclear analysis of fusion energy
systems, and advancing the tools necessary to perform such analysis.  Decades
of experience in fusion nuclear analysis have informed the development of a
suite of approaches for this analysis that span simplified 1-D analysis for
initial system design scoping to high-fidelity 3-D analysis for detailed
assessment of system performance.

During this performance period, two separate program elements were combined
into a single increasingly integrated project.  One program element has
historically focused on the fundamental nuclear performance of components for
a fusion energy system, with particular attention to the design and analysis
of specific first wall and blanket concepts and the development of new
predictive tools for modeling and simulation of those concepts.  This
performance period saw a decline in design and analysis within this program
element, with a shift towards the development of modeling tools and
capability.  The other program element focused on the nuclear analysis of
complete fusion energy design concepts, as part of a larger integrated design
effort.  While there was always substantial collaboration across these two
program elements within \gls{UW-FTI}, they become formally combined in
February 2017.  This report discusses all of this activity as a combined
program.

The combined \gls{UW-FTI} fusion nuclear science and engineering program has
had two primary tasks:
\begin{enumerate}
\item Development of predictive software tools and data for nuclear analysis
  of fusion energy systems, including:
  \begin{enumerate}
  \item Performance improvements for existing capability,
  \item Development of new capabilities,
  \item Software validation, and
  \item Nuclear data assessment.
  \end{enumerate}
\item Nuclear analysis of complete fusion energy systems as part of an
  integrated design effort, including:
  \begin{enumerate}
  \item Initial radial build of system,
  \item Detailed nuclear analysis, and
  \item Thermo-mechanical and electromagnetic analysis of \glspl{PFC}
  \end{enumerate}
\end{enumerate}

The \gls{UW-FTI} team has long been the primary institution for educating and
training new scientists in the field of fusion neutronics, with graduates
working in fusion and related disciplines throughout the \gls{USDOE} laboratory
system, academia and industry.  During this performance period, one new PhD
was granted, and six additional PhD students have been supported in part by
this program.

Finally, the \gls{UW-FTI} team represents the US fusion neutronics community
in national and international working groups on nuclear data, and in
international working groups on fusion neutronics and fusion radioactive waste
management.


\section{Development of Improved Predictive Capability}



For many years, this program has supported the development of a unique
CAD-based capability for neutron and photon transport in fusion energy
systems, known as the \gls{DAGMC} toolkit.  As the fundamental capability for
radiation transport is considered to be well established, a number of
improvements were made to both the performance of the core capability, and
larger workflows that are based on this fundamental capability.

\subsection{\gls{DAGMC} Performance Improvements}

Over nearly a decade of both the development of \gls{DAGMC} and its use in
nuclear analysis, its performance has been assessed in various ways.  The
simplest assessment is to compare the performance of native MCNP using a
native geometry representation to the performance of DAG-MCNP using CAD
geometry representation that is derived directly from the native
representation.  \gls{DAGMC} has been measured to be as fast as \textapprox 3x
slower and as slow as \textapprox 10x slower.  Notably, such a comparison is
only possible when a native MCNP geometry representation already exists.
Arguably, this is not the target use case for \gls{DAGMC} that is designed to
allow analysts to avoid the person-months of effort that are required to
generate such a complex native MCNP representation.  Nevertheless, is is a
compelling motivator for performance improvements.  Three types of performance
improvement have been examined and are at different stages of implementation.

\glspl{SDF} seek to improve performance by pre-computing the distance to the
nearest boundary from a grid of points within a select number of volumes of
the geometry.  Such a grid can be used during transport to avoid an expensive
computation of the exact distance to a boundary in any specific direction when
it is already known that a collision will occur at a distance that is shorter
than this precomputed nearest boundary.  If the error in the precomputed
distance is well-enough characterized, it can be a reliable way to avoid the
more expensive tests.

A complete implementation of this capability includes tools to assess which
volumes in a geometry will benefit from the initial computational cost and
ongoing memory cost of precomputing and storing this information.  In
particular, theoretical analysis has shown that the benefit is a function of
the ratio of the mean free path of radiation transport to the mean chord
length within the volume, with maximum benefit arising when this ratio is
low\cite{shriwise_NET}.

Testing has confirmed this behavior in real geometries, with the conclusion
that only some volumes within a geometry benefit from this treatment.  In
fusion energy systems, the most complex volumes are often formed from the
implicit complement, and therefore are considered vacuums.  In such volumes,
the mean free path is infinite, and the utility of this approach offers no
benefit.

Other ongoing improvements that began during this period are being implemented
to address other aspects of the algorithm, including modifications to the way
that \glspl{BVH} are constructed and traversed with the goal of achieving a
factor of 2 or more improvement in computational performance.  \gls{BVH} tree
construction algorithms can be adaptively selected as the tree become deeper
to respond to the particular arrangement of triangles that are the basis for
the tree.  In addition, cpu-level \gls{SIMD} parallelism can be used to
traverse the trees more quickly.

\subsection{\gls{SDR} Workflow Development}

With tools readily available to predict the so-called prompt responses to
neutron and photon transport throughout complex fusion energy systems during
plasma operation, much of the fusion neutronics community has turned its
attention to accurately predicting the radiation dose levels caused by delayed
photons during shutdown.  This is important for understanding both the
lifetime dose to some sensitive equipment, as well as the potential dose to
workers during maintenance procedures.  The leading methodology for this kind
of analysis is known as the \gls{R2S} method, because it requires two separate
radiation transport steps, coupled through a transmutation
calculation\cite{various_r2s}.  A fully integrated implementation of this
workflow was implemented within the PyNE software system\cite{pyne}.

The \gls{R2S} workflow starts from a single CAD-based \gls{DAGMC} geometry and
a mesh that defines the spatial resolution of the delayed photon source.
These are used to calculate the multi-group neutron flux on that mesh for use
in the activation/transmutation calculation using DAG-MCNP.  The first step of
\gls{R2S} workflow in PyNE uses the geometry and the multi-group neutron
fluxes to automatically write an input file for ALARA to perform the
transmutation calculation.  If the mesh is a non-conformal Cartesian mesh, a
\gls{DAGMC}-based ray tracing method is used to determine the composite
material composition in each mesh voxel for use by ALARA, starting from the
same \gls{DAGMC} geometry.  If a conformal tetrahedral mesh is used, each mesh
element is assumed to containa a single material.

The second step of the \gls{R2S} workflow reads the ALARA output and generates
a spatially distributed multi-group photon source on the same mesh that was
used to record the neutron fluxes.  A different source mesh is generated for
each cooling time, \emph{i.e.} time after shutdown.  A custom source sampling
function is then able to read this mesh-based source, whether Cartesian or
tetrahedral, in the second radiation transport calculation on the original
\gls{DAGMC} geometry.

This particular implementation is innovative in its support for conformal
tetrahedral mesh as well as biasing schemes for efficeintly sampling the
distributed photon source, and has been validated against the primary
experimental data for this kind of analysis\cite{biondo_r2s}.

Two ongoing improvements that began during this period are an extension to
support moving activated sources during maintenance procedures, and a study of
the propagation of statistical error/uncertainty through the \gls{R2S}
workflow.

\subsection{Automated Variance Reduction for \gls{SDR} Analysis}

For over a decade, deterministic radiation transport methods have been used to
determine the variance reduction parameters that allow more accurate Monte
Carlo radiation transport calculations to provide useful answers in a
reasonable computer time.  Although such methods are well-established for the
prompt photons that arise during operational scenarios, the non-linear
relationship between the neutron flux and the delayed photon source of
\gls{R2S} problems has proven challenging.  An extension of the \gls{R2S}
methodology was developed and implemented to overcome this challenge.

The \gls{GT-CADIS} methodology extends the \gls{CADIS} methodology developed
at \gls{ORNL} by implementing a linearization of the transmutation operator
that can be used to couple the adjoint photon flux to the adjoint neutron
source, thus enabling automated variance reduction for \gls{SDR} problems.
The development of this approach required first demonstrating that such a
linearized operator was a sufficiently accurate representation of the
transmutation operator for such problems.  A mathetmatical derivation
identified the \gls{SNILB} criteria which must be met for this to be correct,
and a thorough analysis of a wide array of fusion materials over many
irradiation conditions confirmed that such criteria are indeed met for all
cases of interest.

The \gls{GT-CADIS} methodology was studied in detail on a simplified geometry
designed to demonstrate the benefit of this approach over other possible
approaches for \gls{SDR} problems.  \gls{GT-CADIS} was found to offer a
substantial speedup over the next best choice, \gls{FW-CADIS}.  A single
octant model of a realistic fusion energy system was also used to demosntrate
the utility of this approach.

\subsection{Software Valudation with JET Operational Data}

A collaboration has been established with EURATOM-JET for the neutronics
analysis of JET.  In addition to validating the EU neutronics transport codes
and data, the JET Project, through JET3 - DT Technology EUROfusion Consortium,
would like to widen the scope of JET experiments by also including the
validation of other newly developed transport codes/data used by the US.
Specifically, the collaboration would seek to validate ADVANG, ATTILA, and
DAG-MCNP, including the \gls{R2S} methodology that relies upon it, by using
the experimental data to be provided by JET.

Two different benchmark exercises were devised, one focusing on streaming of
neutrons and another on the shutdown dose rate.  The former is particularly
valuable for the validation of advanced hybrid variance reduction schemes,
such as ADVANTG.  The other is valuable for validation of \gls{SDR}
calculation methodologies, and particularly the \gls{R2S} methodology
developed under this project.

Because the geometric description of the JET facility is already available in
the native MCNP text files, participating in this effort required first
generating a \gls{CAD}-based description.  This was performed using the
MCNP2CAD conversion tool developed in conjunction with the \gls{DAGMC}
toolkit.  These \gls{CAD}-based models have already been used to replicate the
streaming benchmark and are currently being used for the \gls{SDR} benchmark.

\subsection{Nuclear Data for Fusion Applications}

In addition to the tools and workflows described above, robust nuclear
analysis relies on high quality nuclear and atomic data to describe the
interactions of radiation with matter.  Under this project and its
predecessors, we are responsible for monitoring the the development of updated
nuclear data sets and assessing the impacts of those updates on the nuclear
analysis outcomes for various systems.  In addition to attending meetings of
the \gls{CSEWG} and \gls{FENDL} committees, a computational MCNP benchmark is
used to probe the differences in data libraries as they are released.

During this reporting period, the newest release of the \gls{FENDL}, version
3, was made available.  The computational benchmark was used with the
release of \gls{FENDL}-3, showing a modest increase of the neutron flux levels
in the deep penetration regions and a substantial increase in the gas
production in steel components.  The comparison to experimental results showed
good agreement with no substantial differences between FENDL-3.0 and FENDL-2.1
for most responses.  There is a slight trend, however, for an increase of the
fast neutron flux in the shielding experiment and a decrease in the breeder
mock-up experiments. The photon flux spectra measured in the bulk shield and
the tungsten experiments are significantly better reproduced with FENDL-3.0
data. In general, FENDL-3, as compared to FENDL-2.1, shows an improved
performance for fusion neutronics applications. It is thus recommended to ITER
to replace FENDL-2.1 as reference data library for neutronics calculation by
FENDL-3.0.\cite{fischer_benchmarking_2014, bohm_impact_2015}

In addition to these routine tasks, recent work has also begun investigating
how neutronics results depend on the modeled temperature of various
components.  While most systems are modeled at room temperature, the elevated
temperatures of fusion energy systems during normal operation will impact the
results in three distinct ways: (1) reducecd density of materials, (2) Doppler
broadening of cross-sections, and (3) closing of gaps due to thermal
expansion.  Analysis of these effects shows that the Doppler broadening has a
small impact (<1-3\%) on results in fusion energy system, while gap closing
and density reduction can have important effects (4-25\%).


%% Technology/Materials
%%   Sawan paper on comparison of nuclear parameters in MFE,IFE, HFIR?
%% Waste 



\section{Fusion Energy System Studies}

\subsection{Nuclear Analysis}

Nuclear analysis of candidate fusion energy systems has always been an
important aspect of these studies, with particular attention to nuclear
heating, radiation damage, tritium breeding, decay heat removal and waste
management.  Historically, 1-D analysis has been used for scoping potential
design options with 3-D analysis used to probe specific scenarios once a
design has been selected.  Because of the need to generate 3-D models for
nucler analysis, the UW team has become the primary organization for
generating such models.  During this reporting period this kind of analysis
has been completed on the ARIES-ACT-1/2 and FESS-FNSF concepts.  The results
of these nuclear analyses often drive other important design aspects of the
overall system.




\subsection{Thermal-Mechanical Analysis}

\section{Summary}

The \gls{UW-FTI} is pursing an increasingly-integrated research program of
software development for predictive nuclear analysis combined with nuclear
analysis and design of fusion energy systems.  This combination allows the
analysis experience to influence the software development priorities and
allows the analysts to have immediate access to the latest software
developments.

This project and its predecessors have been the primary support for the
development of a leading CAD-based Monte Carlo radiation transport capability
for the US fusion neutronics community, and, until recently\cite{serpentCAD},
the only capability that directly uses high-fidelity CAD-based surface
representation for MC radiation transport in the world.  Continued performance
enhancements draw from a wide body of literature in computer graphics, meshing
and computer science.  Ongoing capability extensions are addressing the
fundamental needs


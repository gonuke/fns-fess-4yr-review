\chapter{Executive Summary}

%% ABSTRACT OR EXECUTIVE SUMMARY: A one page, self-contained program summary
%% written in terms understandable by an educated layperson. The project summary
%% must not exceed one page when printed using standard 8.5” by 11” paper with 1”
%% margins (top, bottom, left, and right) with font not smaller than 11 point.

The fundamental objective of the \gls{UW-FTI}'s fusion nuclear science and
engineering program is based upon providing nuclear analysis of fusion energy
systems, and advancing the tools necessary to perform such analysis.  Decades
of experience in fusion nuclear analysis have informed the development of a
suite of approaches for this analysis that span simplified 1-D analysis for
initial system design scoping to high-fidelity 3-D analysis for detailed
assessment of system performance.

Software enhancements have improved performance and extended capability.
\glspl{SDF} can be pre-computed to avoid expensive ray-firing operations, but
testing has shown that the features that make ray-firing expensive also result
in fewer ray-firing operations.  Continuing efforts employ low level
parallelism to further accelerate tree-traversal and generate those trees with
more fidelity.  Tools are being assembled into robust workflows that
facilitate \gls{SDR} analysis.  The \gls{R2S} methodology is now available as part of
the PyNE package, to coordinate the Monte Carlo radiation transport
simulations with the geometry discretization and activation simulations needed
to generate the distributed photon source.  Automated variance reduction
methods are vital for achieving reasonable runtimes for these complex
problems, and a new methodology was developed to extend the previous
capability for \gls{SDR} problems. The \gls{SDR} workflow is being benchmarked
against experimental measurements from JET.

These tools and workflows are being demonstrated by application to the
analysis/design of candidate fusion energy systems as part of the \gls{FESS},
giving an opportunity to identify areas for improvement and enhancement.  In
addition to calculating specific engineering quantities that inform the design
of each concept, some forms of analysis provide insight into aspects of the
design of any system.  New 3-D analysis capability has been used to study the
effects of engineering-scale heterogeneity of the model on the simulated
\gls{TBR}. This provides insight into the degree of approximation that is
inherent in the 1-D simulations often used for scoping, and also the penalty
in tritium breeding that is associated with particular features such as ducts
and ports in the first wall.  Three-dimensional analysis has been used on
ARIES-ACT-1 and -2, as well as the \gls{FNSF} to examine the distribution of
nuclear heating and radiation damage throughout the system.  Ongoing work is
exploring the impact of liquid metal \glspl{PFC} on the nuclear response of
the system.

Thermo-structural and electromagnetic analysis have been conducted for both
normal and off-normal conditions.  An investigation of solid tungsten surfaces
concluded that all-tungsten \glspl{PFC} are probably viable, but ELM control
is likely required and improved tungsten-based alloys would be beneficial.
The structural response to electromagnetic loads induced by a disruption was
evaluated for the \gls{FNSF}, showing that disruptions appear to be survivable
without significant damage.  As with the nuclear analysis, understanding the
thermo-mechanical and electromagnetic responses to liquid metal \glspl{PFC} is
of current interest, requiring the addition of evaporation and boiling to the
modeling capability.  Electromagnetic effects can be substantial, depending on
design specifics, possibly resulting in liquid metals being emptied from their
trays in the most extreme conditions.

The \gls{UW-FTI} program provides national and international leadership,
representing the US fusion community in two IEA subcommittees, in the IAEA's
fusion nuclear data activities, and in the \gls{CSEWG} that oversees the
revision of the ENDF nuclear data set.  The \gls{UW-FTI} program also educates
and prepares a robust cohort of new scientists and engineers to ensure
continuity in this field that is important for fusion energy system design, as
has been demonstrated in ITER.

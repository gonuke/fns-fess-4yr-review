\chapter{Budget Summary \& Justification}

%% BUDGET SUMMARY & JUSTIFICATION: This information is used to assess the scale
%% and quality of usage of FES provided budgetary resources available to the
%% program during the assessment period.  This section should provide a
%% high-level overview and justification of the base programs typical annual
%% budget in reference to objectives, tasks, and accomplishments described in the
%% program narrative.  Exact organization is not prescribed, but should include a
%% general breakdown into important items such as personnel support, materials
%% and supplies, facility/rental fees, travel, indirect costs, etc.  Programs
%% should also list any major equipment purchases made during the performance
%% period.

\begin{center}
\begin{tabular}{|c|c|c|c|c|}\hline
                        & FY14 & FY15 & FY16 & FY17 \\\hline\hline
  \multicolumn{5}{|l|}{Senior Personnel }\\\hline
  Wilson                &      -    &     -     &     -     &   \$5,515 \\\hline
  Sawan                 &  \$36,830 &  \$50,545 &  \$36,627 &     -     \\\hline
  El-Guebaly            &  \$56,425 &  \$93,841 &  \$48,137 &  \$10,975 \\\hline
  Blanchard             &  \$21,593 &      -    &      -    &    -      \\\hline
  Bohm                  &  \$11,890 &  \$14,481 &  \$21,415 &  \$41,458 \\\hline
  Davis                 &  \$18,454 &   \$5,645 &  \$20,066 &  \$33,566 \\\hline\hline
  Other Professionals   &  \$24,116 &  \$44,033 &  \$58,843 &  \$43,927 \\\hline\hline
  Graduate Students     &  \$10,577 &  \$23,099 &  \$30,415 &  \$53,466 \\\hline\hline
  \emph{Total Salaries} & \emph{\$179,886} & \emph{\$231,643} & \emph{\$215,502} & \emph{\$187,906} \\\hline\hline
  Fringe Benefits       &  \$60,826 &  \$77,988 &  \$75,226 &  \$63,636 \\\hline\hline
  Tuition Remission     &   \$1,768 &   \$9,111 &   \$8,000 &  \$26,667 \\\hline\hline
  Travel                &  \$27,502 &  \$18,414 &  \$25,800 &  \$24,584 \\\hline\hline
  Software Licenses     &   \$6,359 &   \$6,150 &   \$8,050 &  \$12,610 \\\hline\hline
  Other Expenses        &   \$1,233 &   \$3,159 &   \$1,886 &   \$1,515 \\\hline\hline
  Indirect Costs        & \$139,257 & \$170,290 & \$167,531 & \$152,432 \\\hline\hline
  \textbf{Total}        & \textbf{\$416,830} & \textbf{\$516,755} & \textbf{\$501,995} & \textbf{\$469,350} \\\hline
\end{tabular}
\end{center}

\noindent The budget for this project (and those of its predecessors) are
dominated by personnel costs, with salaries, fringe benefits and tuition
remission comprising nearly 90\% of the costs, on average.  Travel costs range
from 5-10\% with the remainder composed of publication costs, software
licenses and other incidentals costs.

\section{Personnel}

\subsection{Senior Personnel}
\paragraph{Wilson} is the PI of this project, and was Co-PI of both
prior projects.  His duties are to oversee the management of the project in
general, and supervise the graduate students' research.  He is the technical
leader of all code development efforts.  Wilson also serves as the US
representative to the IEA Sub-committee on Fusion Neutronics.

\paragraph{Sawan} was PI of one of the prior projects, and oversaw the
management of that project and technical direction of some of its tasks.  He
retired during FY16.

\paragraph{El-Guebaly} was PI of the other prior project organized around the
\gls{FESS} program, and oversaw the management of that project and the
technical direction of its nuclear design and analysis tasks, as well as
performing much of the work scope independently.  She also serves as the US
representative to the IEA Sub-committee on Fusion Radioactive Waste Management.
She officially retired during FY16, but continues to fill this international
role on a contract basis.

\paragraph{Blanchard} provides technical direction for the thermal, structural
and electromagnetic analysis of the \gls{FESS} design efforts.  He supervises
a professional engineering staff member who performs much of the analysis.

\paragraph{Bohm} is Co-PI of the current project, and leads all current
nuclear analysis efforts, including benchmarking against experimental methods
at JET and support for the \gls{FESS}, as well as being the lead on 

\paragraph{Davis} was Co-PI of the current project until he resigned his
position at the UW-Madison at the end of FY17.  Until then, he was a leader in
the software development tasks and supported 3-D analysis for the \gls{FESS}
effort.

\subsection{Other Professionals}

Over the performance period, the project has employed a number of other
professionals to support the project activities.  Specific roles include CAD
engineering, mechanical engineering analysis, and software developer.  The
total effort across all of these roles averages about 0.6 FTE/year.

\subsection{Graduate Students}

The \gls{UW-FTI} educates more new scientists and engineers in fusion
neutronics than any other program in the US.  Some of these students continue
to work in fusion neutronics and closely related fields with others seek other
opportunities in a broader nuclear engineering context. During the period of
performance, eight different PhD students have been engaged with supporting
the software development and systems analysis efforts.  These students are
highly leveraged by other projects and external fellowships such that the
average support from this project is only about 1.5 FTE/yr.

\begin{table*}[h]
  \centering
  \caption{Sources of Support for Graduate Students}
  \begin{tabular}{|c|c|c|c|c|}\hline
    \# of PhD Students & This Project & Other Extramural  & External    & Internationally- \\
    &              & Research Project  & Fellowships & supported Visitors    \\\hline
    1 & X &   &   &   \\\hline
    2 & X & X &   &   \\\hline
    1 & X & X & X &   \\\hline
    2 &   &   & X &   \\\hline
    1 &   &   &   & X \\\hline
  \end{tabular}
\end{table*}

\paragraph{Tuition Remission:} Graduate student support automatically results
in tuition charges of \$12,000/FTE/yr, distributed through the academic year.
Because of the timing of student support, the average support over the
performance period is just less than 1 FTE/yr.

\section{Travel}

This project and its predecessors include funding for both domestic and
international travel to support participation in technical conferences and
collaboration meetings for the \gls{FESS} program. \gls{UW-FTI} staff and
students regularly present the work developed in this project at:
\begin{itemize}
\item International Symposium of Fusion Nuclear Technology (rotating US/international),
\item ANS Topical on Fusion Energy (US),
\item ANS Mathematics and Computation Topical Meeting (rotating US/international),
\item ANS Radiation Protection and Shielding Topical Meeting (rotating
  US/international),
\item IEEE Symposium on Fusion Energy (US), and
\item Symposium on Fusion Technology (EU).
\end{itemize}

Some additional travel support is provided to fulfill the roles as the US
representatives to IEA sub-committees, FENDL data committee, and the US
\gls{CSEWG} data working group.

\section{Other Direct Costs}

Nearly all the remaining costs are due to software licenses for:
\begin{itemize}
\item ANSYS - used for thermo-structural and electromagnetic analysis
\item Trelis - used for DAGMC geometry preparation
\item SpaceClaim - used for DAGMC geometry preparation
\end{itemize}


\section{Introduction}

The fundamental objective of the \gls{UW-FTI}'s fusion nuclear science and
engineering program is based upon providing nuclear analysis of fusion energy
systems, and advancing the tools necessary to perform such analysis.  Decades
of experience in fusion nuclear analysis have informed the development of a
suite of approaches for this analysis that span simplified 1-D analysis for
initial system design scoping to high-fidelity 3-D analysis for detailed
assessemnt of system performance.

During this performance period, two separate program elements were combined
into a single increasingly integrated project.  One program element has
historically focused on the fundamental nuclear performance of components for
a fusion energy system, with particular attention to the design and analysis
of specific first wall and blanket concepts and the development of new
predictive tools for modeling and simulation of those concepts.  This
performance period saw a decline in design and analysis within this program
element, with a shift towards the development of modeling tools and
capbaility.  The other program element focused on the nuclear analysis of
complete fusion energy design concepts, as part of a larger integrated design
effort.  While there was always substantial collaboration across these two
program elements within \gls{UW-FTI}, they become formally combined in
February 2017.  This report discusses all of this activity as a combined
program.

The combined \gls{UW-FTI} fusion nuclear science and engineering has had two
primary tasks:
\begin{enumerate}
\item Development of predictive software tools and data for nuclear analysis
  of fusion energy systems, including:
  \begin{enumerate}
  \item Performance improvements for existing capability,
  \item Development of new capabilities,
  \item Software validation, and
  \item Nuclear data assessment.
  \end{enumerate}
\item Nuclear analysis of complete fusion energy systems as part of an
  integrated design effort, including:
  \begin{enumerate}
  \item Initial radial build of system,
  \item Detailed nuclear analysis, and
  \item Thermo-mechanical and electromagnetic analysis of in-vessel components
  \end{enumerate}
\end{enumerate}

The \gls{UW-FTI} team has long been the primary institution for educating and
training new scientists in the field of fusion neutronics, with graduates
working in fusion and related discplines throughout the \gls{USDOE} laboratory
system, academia and industry.  During this performance period, one new PhD
was granted, and six additional PhD students have been supported in part by
this program.

Finally, the \gls{UW-FTI} team represents the US fusion neutronics community
in national and international working groups on nuclear data, and in
international working groups on fusion neutronics and fusion radioactive waste
management.


\section{Summary}

The \gls{UW-FTI} is pursing an increasingly-integrated research program of
software development for predictive nuclear analysis combined with nuclear
analysis and design of fusion energy systems.  This combination allows the
analysis experience to influence the software development priorities and
allows the analysts to have immediate access to the latest software
developments.

This project and its predecessors have been the primary support for the
development of a leading CAD-based Monte Carlo radiation transport capability
for the US fusion neutronics community, and, until
recently\citeref{talamo_serpent_2018}, the only capability that directly uses
high-fidelity CAD-based surface representation for MC radiation transport in
the world.  Continued performance enhancements draw from a wide body of
literature in computer graphics, meshing and computer science.  Ongoing
capability extensions are addressing the fundamental needs of analysts to
automate the process of performing more complex multi-step analysis, including
variance reduction schemes to accelerate such simulations.

In addition to performing routine design tasks, novel analyses have been
pursued during the development of different fusion energy concepts under the
\gls{FESS} program.  One-dimensional analysis is used to develop a candidate
radial build that considers a variety of performance metrics, including
nuclear heating, radiation damage, tritium breeding ratio, and activation.
This is then translated into a 3-D CAD model for more detailed analysis.
Recent work had shown that 1-D analysis is conservative for activation, but
can grossly overestimate the overall \gls{TBR} without accounting for local
heterogeneities and access ports.  High heat flux components have been
analyzed for their ability to survive both the thermal and electromagnetic
shock of various plasma disruptions.  Work is underway to perform similar
analyses for liquid metal surfaces.
